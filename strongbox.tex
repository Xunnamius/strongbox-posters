%%%%%%%%%%%%%%%%%%%%%%%%%%%%%%%%%%%%%%%%%
% a0poster Portrait Poster
% LaTeX Template
% Version 1.0 (22/06/13)
%
% The a0poster class was created by:
% Gerlinde Kettl and Matthias Weiser (tex@kettl.de)
% 
% This template has been downloaded from:
% http://www.LaTeXTemplates.com
%
% License:
% CC BY-NC-SA 3.0 (http://creativecommons.org/licenses/by-nc-sa/3.0/)
%
%%%%%%%%%%%%%%%%%%%%%%%%%%%%%%%%%%%%%%%%%

%----------------------------------------------------------------------------------------
%	PACKAGES AND OTHER DOCUMENT CONFIGURATIONS
%----------------------------------------------------------------------------------------

\documentclass[a0,portrait]{a0poster}

\usepackage{multicol} % This is so we can have multiple columns of text side-by-side
\columnsep=100pt % This is the amount of white space between the columns in the poster
\columnseprule=1pt % This is the thickness of the black line between the columns in the poster

\usepackage[svgnames]{xcolor} % Specify colors by their 'svgnames', for a full list of all colors available see here: http://www.latextemplates.com/svgnames-colors

\usepackage{times} % Use the times font
%\usepackage{palatino} % Uncomment to use the Palatino font

\usepackage{graphicx} % Required for including images
\graphicspath{{figures/}} % Location of the graphics files
\usepackage{booktabs} % Top and bottom rules for table
\usepackage[font=small,labelfont=bf]{caption} % Required for specifying captions to tables and figures
\usepackage{amsfonts, amsmath, amsthm, amssymb} % For math fonts, symbols and environments
\usepackage{wrapfig} % Allows wrapping text around tables and figures

\begin{document}

%----------------------------------------------------------------------------------------
%	POSTER HEADER 
%----------------------------------------------------------------------------------------

% The header is divided into two boxes:
% The first is 75% wide and houses the title, subtitle, names, university/organization and contact information
% The second is 25% wide and houses a logo for your university/organization or a photo of you
% The widths of these boxes can be easily edited to accommodate your content as you see fit

\begin{minipage}[b]{\linewidth}

\includegraphics[height=5cm]{figures/uchicago_logo.png}
\vspace{0.1cm}
\begin{center}
\noindent\makebox[\linewidth]{\rule{0.9\paperwidth}{0.4pt}}
\vspace{0.1cm}
\end{center}

\veryHuge \color{NavyBlue} \textbf{Strongbox: Fast Secure Storage for Mobile Devices} \color{Black}\\[0.5cm] % Title
% \Huge\textit{An Exploration of Complexity}\\[2cm] % Subtitle
\huge \textbf{Bernard Dickens, Ariel Feldman, Haryadi Gunawi, Henry Hoffmann}\\[0.25cm] % Author(s)
\Large University of Chicago %//[0.25cm] % University/organization
% \Large \texttt{bd3@cs.uchicago.edu}\\
\end{minipage}

\vspace{1cm} % A bit of extra whitespace between the header and poster content

%----------------------------------------------------------------------------------------

\begin{multicols}{2} % This is how many columns your poster will be broken into, a portrait poster is generally split into 2 columns

%----------------------------------------------------------------------------------------
%	ABSTRACT
%----------------------------------------------------------------------------------------

\color{Navy} % Navy color for the abstract

\begin{abstract}

Full disk encryption (FDE) is especially important for mobile devices because
they both contain large amounts of sensitive data and are easily lost or stolen.
Yet, the conventional approach to FDE, AES in XTS mode, is 3--5x slower than
unencrypted storage. Authenticated encryption based on stream ciphers like
ChaCha20 is already used as a faster alternative to AES in other contexts, such
as HTTPS, but the conventional wisdom is that stream ciphers are a unsuitable
for FDE. Used naively in disk encryption, stream ciphers are vulnerable to
many-time pad attacks and rollback attacks, and mitigating these attacks with
on-disk metadata is generally believed to ruin performance.

In this paper, we argue that recent developments in mobile devices invalidate
this assumption and make it possible to use fast stream ciphers for disk
encryption. Modern mobile devices rely on NAND-flash storage with a Flash
Translation Layer (FTL), which functions very similarly to a Log-structured File
System (LFS), and include trusted hardware such as Trusted Execution
Environments (TEEs) and secure storage areas. Leveraging these two trends, we
propose StrongBox, a stream cipher-based FDE layer that is a drop-in replacement
for dm-crypt, the standard Linux disk encryption module based on AES-XTS.
StrongBox introduces a system design and on-disk data structures that exploit
LFS's lack of overwrites to avoid costly rekeying and a counter stored in
trusted hardware to implement rollback protection. We implement StrongBox on an
ARM big.LITTLE mobile processor and test its performance under the F2FS LFS.

\end{abstract}

%----------------------------------------------------------------------------------------
%	INTRODUCTION
%----------------------------------------------------------------------------------------

\color{SaddleBrown} % SaddleBrown color for the introduction

\section*{Motivation}

Aliquam non lacus dolor, \textit{a aliquam quam} \cite{Smith:2012qr}. Cum sociis natoque penatibus et magnis dis parturient montes, nascetur ridiculus mus. Nulla in nibh mauris. Donec vel ligula nisi, a lacinia arcu. Sed mi dui, malesuada vel consectetur et, egestas porta nisi. Sed eleifend pharetra dolor, et dapibus est vulputate eu. \textbf{Integer faucibus elementum felis vitae fringilla.} In hac habitasse platea dictumst. Duis tristique rutrum nisl, nec vulputate elit porta ut. Donec sodales sollicitudin turpis sed convallis. Etiam mauris ligula, blandit adipiscing condimentum eu, dapibus pellentesque risus.

\textit{Aliquam auctor}, metus id ultrices porta, risus enim cursus sapien, quis iaculis sapien tortor sed odio. Mauris ante orci, euismod vitae tincidunt eu, porta ut neque. Aenean sapien est, viverra vel lacinia nec, venenatis eu nulla. Maecenas ut nunc nibh, et tempus libero. Aenean vitae risus ante. Pellentesque condimentum dui. Etiam sagittis purus non tellus tempor volutpat. Donec et dui non massa tristique adipiscing.

%----------------------------------------------------------------------------------------
%	OBJECTIVES
%----------------------------------------------------------------------------------------

\color{DarkSlateGray} % DarkSlateGray color for the rest of the content

\section*{Design and Implementation}

\begin{enumerate}
\item Lorem ipsum dolor sit amet, consectetur.
\item Nullam at mi nisl. Vestibulum est purus, ultricies cursus volutpat sit amet, vestibulum eu.
\item Praesent tortor libero, vulputate quis elementum a, iaculis.
\item Phasellus a quam mauris, non varius mauris. Fusce tristique, enim tempor varius porta, elit purus commodo velit, pretium mattis ligula nisl nec ante.
\item Ut adipiscing accumsan sapien, sit amet pretium.
\item Estibulum est purus, ultricies cursus volutpat
\item Nullam at mi nisl. Vestibulum est purus, ultricies cursus volutpat sit amet, vestibulum eu.
\item Praesent tortor libero, vulputate quis elementum a, iaculis.
\end{enumerate}

%----------------------------------------------------------------------------------------
%	MATERIALS AND METHODS
%----------------------------------------------------------------------------------------

\section*{StrongBox vs Dm-crypt under F2FS}

Fusce magna risus, molestie ut porttitor in, consectetur sed mi. Vestibulum ante ipsum primis in faucibus orci luctus et ultrices posuere cubilia Curae; Pellentesque consectetur blandit pellentesque. Sed odio justo, viverra nec porttitor vel, lacinia a nunc. Suspendisse pulvinar euismod arcu, sit amet accumsan enim fermentum quis. In id mauris ut dui feugiat egestas. Vestibulum ac turpis lacinia nisl commodo sagittis eget sit amet sapien.

%----------------------------------------------------------------------------------------
%	CONCLUSIONS
%----------------------------------------------------------------------------------------

\color{SaddleBrown} % SaddleBrown color for the conclusions to make them stand out

\section*{Conclusion}

\begin{itemize}

\item The conventional wisdom: securing data at rest requires one must pay the
large performance overhead of encryption with the AES-XTS block cipher instead
of using a stream cipher.

\item The proliferation of NAND-flash FTL/LFS and secure hardware on
modern/mobile devices overturn the conventional wisdom, making it practical to
use a stream ciphers to secure data at rest.

\item We propose StrongBox, a stream cipher-based FDE layer and drop-in
replacement for dm-crypt. StrongBox exploits LFS’s lack of overwrites and the
availability of trusted hardware to overcome the limitations of stream ciphers.

\item Our results show that under F2FS, StrongBox provides upwards of
$2\times$ improvement on read performance and $1.21\times$ improvement on write
performance over a standard dm-crypt configuration.

\end{itemize}

\color{DarkSlateGray} % Set the color back to DarkSlateGray for the rest of the content

%----------------------------------------------------------------------------------------
%	REFERENCES
%----------------------------------------------------------------------------------------

\nocite{*} % Print all references regardless of whether they were cited in the poster or not
\bibliographystyle{plain} % Plain referencing style
\bibliography{strongbox} % Use the example bibliography file sample.bib

%----------------------------------------------------------------------------------------

\end{multicols}

\begin{minipage}[b]{\linewidth}
\vspace{1cm}
\noindent\makebox[\linewidth]{\rule{0.9\paperwidth}{0.4pt}}
StrongBox source is available on GitHub @ \texttt{https://github.com/ananonrepo2/StrongBox}
\end{minipage}

\end{document}
